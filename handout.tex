\documentclass[a4paper]{article}
\usepackage[utf8]{inputenc}
\title{Ziegenproblem-Handout}
\author{Der Pöblizist }
\usepackage{graphicx}
\usepackage{amsmath}
\usepackage{tikz}
\usepackage{hyperref}
\usetikzlibrary{trees}
\begin{document}
\maketitle
\section{Was ist das Ziegenproblem?}
Das Ziegenproblem ist ein Problem der Mathematik, bei dem diese Regeln gelten:
\begin{itemize}
    \item Ein Auto und zwei Ziegen werden zufällig auf drei Tore verteilt.
    Zu Beginn des Spiels sind alle Tore verschlossen, sodass Auto und Ziegen nicht sichtbar sind.
    Der Kandidat, dem die Position des Autos völlig unbekannt ist, wählt ein Tor aus, das aber vorerst verschlossen bleibt.
    \item Fall A: Hat der Kandidat ein Tor gewählt, öffnet der Moderator ein Tor, hinter dem sich eine Ziege befindet.
    \item Der Moderator bietet dem Kandidaten an, seine Entscheidung zu überdenken und das andere ungeöffnete Tor zu wählen.
    \item Das vom Kandidaten letztlich gewählte Tor wird geöffnet, und er erhält das Auto, falls es sich hinter diesem Tor befindet.
\end{itemize}
\section{unerwartete Wahrscheinlichkeit}
Entgegen der Erwartung ist der Wechsel immer vorzuziehen, mit Wechsel beträgt die Autowahrscheinlichkeit 66,6 \%, ohne nur 33,3\%.
\section{Entscheidungsbaum}
\scalebox{0.75}{
% Set the overall layout of the tree
\tikzstyle{level 1}=[level distance=3.5cm, sibling distance=3.5cm]
\tikzstyle{level 2}=[level distance=3.5cm, sibling distance=2cm]

% Define styles for bags and leafs
\tikzstyle{bag} = [text width=4em, text centered]
\tikzstyle{end} = [circle, minimum width=3pt,fill, inner sep=0pt]
\scalebox{0.75}{
\begin{tikzpicture}[grow=right, sloped]
\node[bag] {Wahl}
    child {
        node[bag] {Ziege 1}% This is the first of three "Bag 2"
        child {
                node[bag,fill=green!30!] {Wechsel}
                child {
                    node[bag,fill=green!30!] {Auto}
                }
            }
            child {
                node[bag,fill=yellow!30!] {Kein Wechsel}
                child{
                    node[bag,fill=yellow!30!] {Ziege 1}
                }
            }
    }
    child {
        node[bag] {Ziege 2}% This is the first of three "Bag 2"
        child {
                node[bag,fill=green!30!] {Wechsel}
                child {
                    node[bag,fill=green!30!] {Auto}
                }
            }
            child {
                node[bag,fill=yellow!30!] {Kein Wechsel}
                child{
                    node[bag,fill=yellow!30!] {Ziege 2}
                }
            }
    }
        child {
            node[bag] {Auto}
            child {
                    node[bag,fill=green!30!] {Wechsel}
                    child {
                        node[bag,fill=green!30!] {Ziege}
                    }
                }
                child {
                    node[bag,fill=yellow!30!] {Kein Wechsel}
                    child{
                        node[bag,fill=yellow!30!] {Auto}
                    }
                }
    };
\end{tikzpicture}
}
}
\large
\section{Ausgedrückt in Ergebnismengen:}
		\begin{align*}
		\text{Ergebnisraum:}\\
		\Omega &= \{Ziege1, Ziege2, Auto\}; \; |\Omega|= 3\\
		\text{Ergebnismenge mit Wechsel:}\\
		\Omega_{Wechsel} &= \{Ziege1, Ziege2\};\; |\Omega_{Wechsel}|=2\\
		\text{Ergebnismenge ohne Wechsel:}\\
		\Omega_{ohne Wechsel} &= \{ Auto \};\;|\Omega_{ohne Wechsel}|= 1\\
		\text{Daraus folgt: }\\
		P_{Wechsel} &= \frac{ |\Omega_{Wechsel}|}{|\Omega|}= \frac{2}{3}\\
		P_{ohne Wechsel} &= \frac{|\Omega_{ohne Wechsel}|}{|\Omega|}= \frac{1}{3}
		\end{align*}
\section{Ausgedrückt als abhängige Wahrscheinlichkeit:}
	$w_a$, $w_z$: Es wurde anfangs das Auto, bzw. eine Ziege gewählt.\\
	$W_a$, $W_z$: Es wurde nach der Umwahl das Auto, bzw. die Ziege gewählt.\\
	Nach dem Satz der totalen Wahrscheinlichkeit:
	\begin{align*}
	P(a) &= P(W_a \land w_z) + P(W_a \land w_a) \\&= P(w_z) * P(w_z|W_a) + P(w_a) * P(W_a|w_a)\\
	&= \frac{2}{3} * 1 + \frac{1}{3} * 0 = \frac{2}{3}
	\end{align*}
\section{Quellen}
		\url{https://de.wikipedia.org/wiki/Hausziege#/media/Datei:Hausziege_04.jpg}\\
		\url{https://thumbs.dreamstime.com/b/drei-t\%C3\%BCren-1875644.jpg}\\
		\url{https://www.grin.com/document/214288}\\
		\url{https://www.pedocs.de/volltexte/2013/5807/pdf/UntWiss_2004_1_Krauss_Atmaca_Schueler_Einsicht.pdf}
\section{Gesamtes Projekt}
\url{https://github.com/kuseler/mathe-praesi}
\end{document}
